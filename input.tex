% Markdown to \LaTeX \linebreak シンプルに文章を書くために
% catatsuy
% 


# 書き方

最初にタイトルや名前などを書いておく.

箇条書きなどとてもシンプルに書ける.

* これだけで
* 箇条書きができますよ
* あいうえお

`gcc` とかとてもシンプルにかける.これはすごく嬉しい.

    #include <stdio.h>
    int main() {
      printf("hello, world\n");
      return 0;
    }

簡単なソースコードならこれで十分.

# 使い方

`header.tex` に読み込みたいパッケージや余白設定などを書いておく.

オプションなど長いので `Makefile` を作ること推奨.

`make tolatex` で `pandoc` で Markdown の文章を \LaTeX 形式に出来る.

`make topdf` で `pdf` ファイルを出力できる.

`make` もしくは `make all` で上記の 2 つを同時に実行できる.

`pandoc` を `pandoc.my` という名前のリネームしているので自分用に直して利用してください.

