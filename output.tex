\documentclass[12Q,papersize]{jsarticle}
\usepackage[T1]{fontenc}
\usepackage[utf8]{inputenc}
\usepackage{lmodern}
\usepackage{amssymb,amsmath}
\usepackage{calc}
\setlength{\textheight}{48\baselineskip+\topskip}
\setlength{\topmargin}{(297truemm-\textheight)/2-1truein-2truecm}
\setlength{\textwidth}{55zw}
\setlength{\oddsidemargin}{(210truemm-\textwidth)/2-1truein}
\usepackage[sc]{mathpazo}
\usepackage[scaled]{helvet}
\usepackage[scaled]{beramono}
\usepackage[deluxe,expert]{otf}
\usepackage{textcomp,okumacro}

\title{Markdown to \LaTeX \linebreak シンプルに文章を書くために}
\author{catatsuy}
\date{}

\begin{document}
\maketitle

\section{書き方}

最初にタイトルや名前などを書いておく.

箇条書きなどとてもシンプルに書ける.

\begin{itemize}
\itemsep1pt\parskip0pt\parsep0pt
\item
  これだけで
\item
  箇条書きができますよ
\item
  あいうえお
\end{itemize}

\verb`gcc` とかとてもシンプルにかける.これはすごく嬉しい.

\begin{verbatim}
#include <stdio.h>
int main() {
  printf("hello, world\n");
  return 0;
}
\end{verbatim}

簡単なソースコードならこれで十分.

\section{使い方}

\verb`header.tex` に読み込みたいパッケージや余白設定などを書いておく.

オプションなど長いので \verb`Makefile` を作ること推奨.

\verb`make tolatex` で \verb`pandoc` で Markdown の文章を
\LaTeX 形式に出来る.

\verb`make topdf` で \verb`pdf` ファイルを出力できる.

\verb`make` もしくは \verb`make all` で上記の 2 つを同時に実行できる.

\verb`pandoc` を \verb`pandoc.my`
という名前のリネームしているので自分用に直して利用してください.

\end{document}
